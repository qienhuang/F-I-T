Large language models (LLMs) have made it easy to generate fluent explanations, hypotheses, and drafts at scale. Yet across disciplines, many AI-assisted research efforts feel productive while failing to accumulate durable knowledge across sessions and teams. This paper argues that the dominant failure mode is structural rather than intellectual: researchers misplace the LLM inside the research loop, confusing conversation with memory and fluency with progress. The result is \emph{fluent stagnation}—rapid local coherence without irreversible commitments, explicit constraints, or falsifiable structure.

We explain why this failure is intrinsic to the conversational medium, why more capable models often intensify it, and what minimal discipline is required to restore accumulation. The proposed remedy is not a new model or tool, but a boundary: AI is returned to the exploration layer, while durable knowledge is forced into external artifacts that can be versioned, constrained, and defeated. The contribution is a negative one—an account of why most AI-assisted research fails by default—and a minimal set of conditions under which AI acceleration can produce knowledge that survives the next conversation.
