Self-referential systems---including advanced machine learning models with self-evaluation, confidence modulation, or meta-learning---exhibit a characteristic failure mode in which internal coherence progressively suppresses external correction, leading to lock-in and catastrophic instability under distributional shift. This paper introduces Controlled Nirvana, a structural safety mechanism that enables non-destructive intervention by temporarily suspending self-referential execution authority. The core mechanism is the Emptiness Window: a bounded interval during which self-referential signals are prevented from governing irreversible actions, while perception, evaluation, and learning remain active. Controlled Nirvana is derived from the Force--Information--Time (FIT) framework {[}1{]} and addresses a structural gap not covered by shutdownability, corrigibility, or related notions of human control and interruption {[}5--8{]}. Rather than proposing a new learning algorithm, we contribute a minimal governance primitive for managing post-grokking risk in self-referential AI systems. The mechanism is positioned within a broader analysis of why internal momentum---the acquisition of self-referential execution authority---represents a distinct and undertheorized category of AI safety risk {[}9{]}.
